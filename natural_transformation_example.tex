\documentclass[]{article}
\usepackage{amsthm}
\usepackage{graphicx}
\usepackage{amsmath}
\usepackage{xcolor}
\usepackage{amssymb}
% ----------------------------------------------------------------
\vfuzz2pt % Don't report over-full v-boxes if over-edge is small
\hfuzz2pt % Don't report over-full h-boxes if over-edge is small
% THEOREMS -------------------------------------------------------
\newtheorem{thm}{Theorem}[section]
\newtheorem{cor}[thm]{Corollary}
\newtheorem{lem}[thm]{Lemma}
\newtheorem{prop}[thm]{Proposition}
\theoremstyle{definition}
\newtheorem{defn}[thm]{Definition}
\theoremstyle{remark}
\newtheorem{rem}[thm]{Remark}
\numberwithin{equation}{section}
% MATH -----------------------------------------------------------
\newcommand{\norm}[1]{\left\Vert#1\right\Vert}
\newcommand{\abs}[1]{\left\vert#1\right\vert}
\newcommand{\set}[1]{\left\{#1\right\}}
\newcommand{\Real}{\mathbb R}
\newcommand{\eps}{\varepsilon}
\newcommand{\To}{\longrightarrow}
\newcommand{\BX}{\mathbf{B}(X)}
\newcommand{\A}{\mathcal{A}}

%opening
\title{Natural Transformation Notes}
\author{Chris and David}

\begin{document}

\maketitle

\begin{abstract}
	These are notes taken in our Functional Programming Study Group
\end{abstract}

\section{Natural Transformations}
	We work out the example of the determinant interpreted as a natural transformation, which is discussed on page 16 of Mac Lane's "Categories for the Working Mathematician."
	
	Recall that a natural transformation is defined in terms of two categories, $(C),\ (B)$, along with two funtors, $S,T: (C)\to (B)$. That is, a natural transformation $\tau$, is really
	\[
		\tau=\tau(S,T,C,B): Ob(C)\to Mor(B).
	\]
	
	We can interpret the determinant as a natural transformation, $$\det = \det (GL_n(\cdot), (\cdot)^\times,(CommAlg),(Grps)).$$
	Here, $GL_n(\cdot),(\cdot)^\times:(CommAlg)\to (Grps)$ are functors. (Note that $n$ is fixed.).
	
	Then for $\mathbb{R}, \mathbb{C}\in Ob(CommAlg)$, let $f:\mathbb{R}\to \mathbb{C}$ be the standard inclusion map (homomorphism). Then
	\begin{align*}
		{\rm det}_\mathbb{R} &= GL_n \mathbb{R}\to \mathbb{R}^\times\\
		{\rm det}_\mathbb{C} &= GL_n \mathbb{C}\to \mathbb{C}^\times
	\end{align*}
	is the value of the determinant natural transformation applied to $\mathbb{R}.$ To see that this is a natural transformation, we have to see that this map commutes with the standard inclusion homomorphism.
	
	
\end{document}
